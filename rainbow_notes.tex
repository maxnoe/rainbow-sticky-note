\documentclass{minimal}
\usepackage[paperwidth=72mm, paperheight=72mm, margin=2.5mm]{geometry}


\usepackage{fontspec}
\setsansfont{TeX Gyre Heros}
\renewcommand\familydefault\sfdefault

\usepackage[american]{babel}

\usepackage[autostyle]{csquotes}

\setlength{\parindent}{0em}
\setlength{\parskip}{0.25\baselineskip}

\usepackage{tcolorbox}

\usepackage{xcolor}
\usepackage[colorlinks, urlcolor=blue!80!black]{hyperref}



\begin{document}
\fontsize{8}{10}\selectfont
\begin{tabular}{@{}c@{}}%
\includegraphics[width=0.3\textwidth]{build/norainbow.pdf}%
\end{tabular}%
\hfill%
\begin{tabular}{@{}c@{}}%
\begin{tcolorbox}[
    colframe=black,
    colback=white,
    fontupper=\raggedright\bfseries,
    width=0.68\textwidth,
    boxrule=2pt,
    sharp corners,
    before skip=0pt,
    before={},
    boxsep=0mm,
    left=1mm, right=1mm, top=1mm, bottom=1mm,
]
    Use of the rainbow colormap threatens the health of you and those around you.\textcolor{blue!80!black}{\textsuperscript{1}}
\end{tcolorbox}
\end{tabular}%

\raggedright%
Please consider using a perceptually uniform colormap, such as:

\includegraphics[width=\linewidth]{build/cmaps.pdf}

Rainbow colormaps distort your data and are not suited for people with color vision deficiencies.\textcolor{blue!80!black}{\textsuperscript{2}}\\
For more info, watch this talk: \href{https://youtu.be/xAoljeRJ3lU}{youtu.be/xAoljeRJ3lU}.


\vspace{\fill}
\fontsize{6}{8}\selectfont
\textcolor{blue!80!black}{\textsuperscript{1}}
\enquote{%
    We show statistically significant [...]
    that a perceptually appropriate color map
    leads to \textbf{fewer diagnostic mistakes} than a rainbow color map.%
} – \href{https://doi.org/10.1109/TVCG.2011.19}{doi:10.1109/TVCG.2011.19}\\
\textcolor{blue!80!black}{\textsuperscript{2}} Affects about 8\,\% of males and 0.5\,\% of females.

\end{document}
